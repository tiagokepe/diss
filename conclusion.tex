\chapter{Conclusion} % (fold)
\label{cha:conclusion}

The initial experiments shown the feedback feature of our solution. This feature
improve more quickly the BA performance that directly impacts the variation of
runtime. Using BA the performance never recide showing up attractive for self-tuning
on hadoop through its convergence.

The sample method on hadoop is an original approach to avoid executing BA on all
data storage. It benefits of the MapReduce paradigm and the key-value model, thus
taking greater advantages on the Hadoop framework.

The context transformation described in \ref{sec:contextTrans} shown the genetic
domain and hadoop domain are consistent for the bijection property. So, the DLS
developed has an intuitive use.

\section{To do}

\begin{itemize}

    \item The front-end(DSL) is not integrated with the others components and some
    rules need to be added to the grammar.

    \item Improve the intregration between AutoConf componet because it has been
    calling as system calls. The task consist in create a library in order to use
    the AutoConf classes.

    \item Yet is missing to test our solution on distributed hadoop. We want to
    run tests increasing machines on the cluster and analyze the performance. So,
    we can analyze what the BA performance on dinamic clusters.

    \item Maybe, improve the BA implementation using heuristics to orient the
    assignment values to the knobs.

\end{itemize}
