\chapter{Initial experiments} % (fold)
\label{cha:experiments}

In this chapter we present initial experiments to show advantages of our proposal.
First we present study case to show the high convergence of BA.

\section{Bacteriological algorithm convergence}

Our first study case aims to show the high convergence of BA. We ran our solution
in Hadoop standalone mode. The test consisted in run the BA with population
size 3, i.e. with three set of knobs per generation, each set of knobs had 10 knobs.
The input files were generated with hadoop job called {\it randomtextwriter} generating
10GB of random text file. The sample percent was of 10 percent of the input, i.e.
1GB of data.

We ran the BA in three rounds, up to three generations, up to six generations and
last one up to twelve generations. The input set of knobs for the first round was
generated randomly, after for the next round the set of knobs were the top three
of the previous round. So, ocurring feedback of the process.

We can realize with the three graphics that passing generations the time never
recede, this characteristic occurs because of the memorization operator. Another
aspect is t 
