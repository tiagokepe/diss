\chapter{Introduction} % (fold)
\label{cha:introduction}

In this chapter we present our motivation and objectives for this work, and we
present the organization of the document.

\section{Motivation}

Currently with the popularity of internet and phenomenon of the social
networks a large amount of data is generated day-to-day. To analyse and
process such quantity of data is needed a big computing power that one
single machine could not analyse such data. To solve it the big companies,
researchers and governments are using distributed computation. To perform
the distributed computation efficiently the data storage must be simple
and so to allow parallel processing. A model that has such features
is the key-value model and the interface with this model is MapReduce
paradigm \cite{Dean:2004}.

MapReduce became the industry de facto standard for parallel processing.
Attractive features such as scalability and reliability motivate many large companies
such as Facebook, Google, Yahoo and research institutes to adopt this new programming
paradigm. Key-value model and MapReduce paradigm are implemented on the framework
Hadoop, an open-source implementation of MapReduce, and these organizations rely
on Hadoop~\cite{White:2009} to process their information. Besides Hadoop, several
other implementations are available: Greenplum MapReduce~\cite{Greenplum:2008},
Aster Data~\cite{Aster:2011}, Nokia Disco~\cite{Mundkur:2011}, 
Microsoft Dryad~\cite{Isard:2007}, among others.

MapReduce has a simplified programming model, where data processing algorithms 
are implemented as instances of two higher-order functions: Map and Reduce. All 
complex issues related to distributed processing, such as scalability, data
distribution and reconciliation, concurrence, fault tolerance, etc., are managed
by the framework. The main complexity that is left to the developer of a 
MapReduce-based application (also called a job) lies in the design decisions made 
to split the application specific algorithm into two higher-order functions. Even
if some decisions may result in a functionally correct application, bad design
choices might also lead to poor resource usage.

Implement jobs on Hadoop is simple, but there are many of knobs to adjust that
depends of the data stored and job running. A good configuration can improve the
job performance and one relevant aspect is that the MapReduce jobs work with
large amounts of data, such fact is the main barrier to find a good configuration.
Therefore a data sample is essencial, but generate a representative and relevant
data sampling is hard and a bad sampling may not represent several aspects
related to the computation in large-scale: efficient resource usage, correct
merge of data, intermediate data, etc.

Hence is very important to adjust the configuration knobs for each job and this
configutarion must be specific for own job. However, according with the cluster
variation, eg. to add or to remove machines, the data insertion or remotion,
may be need to adjust again the job configuration.

Find a good configuration is not so easy and may spend much time. So one way
to automate the job configuration is very useful for users.

\section{Objectives}
Our objective is to propose autoconfiguration of Hadoop, for this we intend to
use an evolutionary algorithm \cite{baudry} to select good configurations
of the jobs. Based in our knowledge the best way to find such configurations is
to run the jobs with its and analyse the performance, but a crucial trouble is
the large amount of data stored that can increase exponentially the test time of
the job. One way to solve this trouble is to create a data sample. We propose
one methodology to implement a data sample using key-value model and MapReduce
paradigm.

\section{Contribution}

We present an original approach to automate Hadoop job configuration, our
implementation is basead in an bacteriological algorithm \cite{baudry} and in
order to use this algorithm we develop a method to obtain data sample on
hadoop cluster. To develop this method we needed to consider a lot of
aspects related the paradigm MapReduce, key-value model and others hadoop
particularities. Our framework has an user interface which have been implementing
with domain specific language ({\bf DSL}), it's a front end for the users and
facilitates the use of the our framework, after ran it the user can obtain the
job configuration resultant, so the users have a tool end-to-end.

The work presented here contributes to the establishment a framework to automate
Hadoop job configuration, through the following proposals:
\begin{itemize}
	\item a interface for users basead on domain specific language;
	\item an algorithm to automate a good choice of jobs configuration;
	\item a method for sampling data on Hadoop clusters.
\end{itemize}

As measure of performance we used the latency time that the job led to
conclude. Furthermore, we intend to use other measures of performance such as
amount of intermidiate data generate, network usage and among others.

\section{Outline}

\begin{itemize}
	\item Chapter \ref{cha:background} introduces the fundamental concepts of the MapReduce framework.
    \item Chapter \ref{cha:bacAlg} presents the bacteriological algorithm.
    \item Chapter \ref{cha:sample} presents the method to generate sampling
    data.
    \item Chapter \ref{cha:dsl} introduces the concepts of the domain specific language.
	\item In chapter \ref{cha:proposal} we presents our framework with all components.
    \item In chapter \ref{cha:experiments} we discussed a case study performed with our solution.
	\item In chapter \ref{cha:conclusion} we conclude our results.
\end{itemize}

%The next section introduces the  fundamental concepts of the MapReduce framework and briefly describes different testing scenarios for MapReduce applications. Chapter~\ref{cha:proposal} introduces two test quality assessment methods, describes methods to data test generation and meta-heuristic search methods. Two data test generators are presented with their results.
%Chapter~\ref{cha:related} discusses related work. 
%Chapter~\ref{cha:conclusion} concludes.

% chapter introduction (end)


