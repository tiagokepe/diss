Atualemente, com a popularidade da internet e o fenômeno das redes sociais uma grande
quantidade de dados é gerada dia à dia. MapReduce aparece como um poderoso paradigma
para analisar e processar tal quantidade de dados. O arcabouço Hadoop implementa
o MapReduce paradigma, no qual uma simples interface está disponível para implementar
programas MapReduce(MR). Entretando, em programas MR desenvolvedores podem configurar
vários paramêtros para otimizar a performance dos recursos disponíveis, mas encontrar
boas configurações consome tempo e uma configuração encontrada em uma execução pode
ser impraticável na próxima vez.

A fim de facilitar e automatizar o ajuste de programas do hadoop, nós propomos um
auto-ajuste baseado em amostragem de dados. Nossa abordagem permite uma boa configuração
considerando os dados armazenados e o programa em questão. Usuários até podem fornecer
suas usuais configurações do programa e então obter uma nova configuração que será
mais apropriada com o estado atual dos dados armazenados e com o cluster do hadoop.
Então os usuários tem uma ferramenta de ponta-a-ponta para automatizar a escolha
de paramêtros para cada programa.
