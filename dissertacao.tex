\documentclass[12pt,a4paper]{ufpr}

% \usepackage[portuges,brazil]{babel}
% \usepackage[portuguese,brazil]{babel}

\usepackage[english]{babel}
\usepackage[utf8]{inputenc}
\usepackage{amssymb,amsmath,amsfonts}
\usepackage{amsmath}
\usepackage{epsfig}
\usepackage{multirow}

\newcommand{\tuple}[1]{\ensuremath{\left \langle #1 \right \rangle }}
\newcommand{\code}[1]{\textbf{#1}}

\usepackage{ifthen,graphicx,color}

\usepackage{verbatim}
%\usepackage[pdfborder={0 0 0},plainpages=false]{hyperref}

\usepackage{listings}
\usepackage{minted}
%\usemintedstyle{trac}

%\usepackage{isolatin1}
\usepackage{amssymb}
\usepackage{subfigure}
\usepackage{caption2}
\usepackage{setspace}
\usepackage{ps-macros}
% \usepackage{psfig}

\usepackage{indentfirst}

\setcounter{secnumdepth}{3}    % n - numero de niveis de subsubsection numeradas
\setcounter{tocdepth}{3}       % coloca ate o nivel n no sumario

\title{KonfigJob: A framework based in bacteriological algorithm for Hadoop
job configuration}
\author{Tiago Rodrigo Kepe}
\advisortitle{Advisor} % ou Orientador
\advisorname{Prof. Dr. Eduardo C. de Almeida}
\advisorplace{Informatics Department, Federal University of Paran�, Brazil}
\city{Curitiba}
\year{2013}

\banca        % nao insira o nome do orientador, ja eh feito automaticamente
{Prof. Dr. Marcos D. Del Fabro}{Informatics Department, Federal University of Paran�}
{Prof. Dr. Gerson Suny�}{Informatics Department, INRIA - University of Nantes, France}
{}{}
{}{}    % se houver um quarto membro na banca, inserir nome e instituicao

\defesa{04 de outubro de 2000} % dia em que foi realizada a defesa da dissertacao


\begin{document}

\makecapaproposta             % cria capa para proposta%
%makecapadissertacao           % cria capa para dissertacao de mestrado %
%makerosto                     % cria folha de rosto para versao final da UFPR %
%\maketermo                     % cria folha com o termo de aprovacao da dissertacao%

%\singlespacing           % espacamento 1 - capa UFPR%
%\onehalfspacing          % espacamento 1/2 %
\doublespacing            % espacamento 2 - UFPR %

\pagestyle{headings}
\pagenumbering{roman}

%\chapter*{Agradecimentos}
%\input{agradecimentos.tex}          % possiu somente o texto

\tableofcontents

%\listoffigures         % se houver mais do que 3 figuras
%\addcontentsline{toc}{chapter}{\MakeUppercase{Lista de Figuras}}
%\newpage

%\listoftables        % se houver mais do que 3 tabelas
%\addcontentsline{toc}{chapter}{\MakeUppercase{Lista de Tabelas}}
%\newpage

\chapter*{Resumo}
\addcontentsline{toc}{chapter}{\MakeUppercase{Resumo}}
Atualemente, com a popularidade da internet e o fenômeno das redes sociais uma grande
quantidade de dados é gerada dia à dia. MapReduce aparece como um poderoso paradigma
para analisar e processar tal quantidade de dados. O arcabouço Hadoop implementa
o MapReduce paradigma, no qual uma simples interface está disponível para implementar
programas MapReduce(MR). Entretando, em programas MR desenvolvedores podem configurar
vários paramêtros para otimizar a performance dos recursos disponíveis, mas encontrar
boas configurações consome tempo e uma configuração encontrada em uma execução pode
ser impraticável na próxima vez.

A fim de facilitar e automatizar o ajuste de programas do hadoop, nós propomos um
auto-ajuste baseado em amostragem de dados. Nossa abordagem permite uma boa configuração
considerando os dados armazenados e o programa em questão. Usuários até podem fornecer
suas usuais configurações do programa e então obter uma nova configuração que será
mais apropriada com o estado atual dos dados armazenados e com o cluster do hadoop.
Então os usuários tem uma ferramenta de ponta-a-ponta para automatizar a escolha
de paramêtros para cada programa.
           % somente o texto
\newpage

\chapter*{Abstract}
\addcontentsline{toc}{chapter}{\MakeUppercase{Abstract}}
% abstract
Currently with the popularity of Internet and phenomenon of the social
networks a large amount of data is generated daily. MapReduce appears as a
powerful paradigm to analyse and process such amount of data. The Hadoop framework
implements the MapReduce paradigm, in which a simple interface is available to
implement MapReduce jobs. However, in MR jobs developers are allowed to setup several
parameters to draw optimal performance from the available resources, but finding
a configuration which best suits to the current state of the cluster and the data
stored is time consuming and a configuration found in an execution may be impracticable
for the next time.

Hadoop cluster administration involve, beyond other tasks, choosing configurations
for each job. In Big Data environments this task is impracticable to be executed
manually because of the huge set of jobs. In order to facilitate and automate tuning
Hadoop jobs, we propose a self-tuning based on data sampling. Our approach allows
to find a job configuration according to the cluster state and the data stored.
Users can provide their usual job configurations then get the new job configuration
that will be more appropriate with the Hadoop current state and the data stored.
So the users have an end-to-end tool to automate the choice of knobs for each job.
        % somente o texto
\newpage


\pagenumbering{arabic}

\chapter{Introduction} % (fold)
\label{cha:introduction}

\section{Motivation}

Currently with the popularity of internet and phenomenon of the social
networks a large amount of data is generated day-to-day. To analyse and
process such quantity of data is needed a big computing power that one
single machine could not analyse such data. To solve it the big companies,
researchers and governments are using distributed computation. To perform
the distributed computation efficiently the data storage must be simple
and so to allow parallel processing. A model that has such features
is the key-value model and the interface with this model is MapReduce
paradigm \cite{Dean:2004}.

MapReduce became the industry de facto standard for parallel processing.
Attractive features such as scalability and reliability motivate many large companies
such as Facebook, Google, Yahoo and research institutes to adopt this new programming
paradigm. Key-value model and MapReduce paradigm are implemented on the framework
Hadoop, an open-source implementation of MapReduce, and these organizations rely
on Hadoop~\cite{White:2009} to process their information. Besides Hadoop, several
other implementations are available: Greenplum MapReduce~\cite{Greenplum:2008},
Aster Data~\cite{Aster:2011}, Nokia Disco~\cite{Mundkur:2011}, 
Microsoft Dryad~\cite{Isard:2007}, among others.

MapReduce has a simplified programming model, where data processing algorithms 
are implemented as instances of two higher-order functions: Map and Reduce. All 
complex issues related to distributed processing, such as scalability, data
distribution and reconciliation, concurrence, fault tolerance, etc., are managed
by the framework. The main complexity that is left to the developer of a 
MapReduce-based application (also called a job) lies in the design decisions made 
to split the application specific algorithm into two higher-order functions. Even
if some decisions may result in a functionally correct application, bad design
choices might also lead to poor resource usage.

Implement jobs on Hadoop is simple, but there are many of knobs to adjust that
depends of the data stored and job running. A good configuration can improve the
job performance and one relevant aspect is that the MapReduce jobs work with
large amounts of data, such fact is the main barrier to find a good configuration.
Therefore a data sample is essencial, but generate a representative and relevant
data sampling is hard and a bad sampling may not represent several aspects
related to the computation in large-scale: efficient resource usage, correct
merge of data, intermediate data, etc.

Hence is very important to adjust the configuration knobs for each job and this
configutarion must be specific for own job. However, according with the cluster
variation, eg. to add or to remove machines, the data insertion or remotion,
may be need to adjust again the job configuration.

Find a good configuration is not so easy and may spend much time. So one way
to automate the job configuration is very useful for users.

\section{Objectives}
Our objective is to propose autoconfiguration of Hadoop, for this we intend to
use an evolutionary algorithm \cite{baudry} to select good configurations
of the jobs. Based in our knowledge the best way to find such configurations is
to run the jobs with its and analyse the performance, but a crucial trouble is
the large amount of data stored that can increase exponentially the test time of
the job. One way to solve this trouble is to create a data sample. We propose
one methodology to implement a data sample using key-value model and MapReduce
paradigm.

\section{Contribution}

We present an original approach to automate Hadoop job configuration, our
implementation is basead in an bacteriological algorithm \cite{baudry} and in
order to use this algorithm we develop a method to obtain data sample on
hadoop cluster. To develop this method we needed to consider a lot of
aspects related the paradigm MapReduce, key-value model and others hadoop
particularities. Our framework has an user interface which have been implementing
with domain specific language ({\bf DSL}), it's a front end for the users and
facilitates the use of the our framework, after ran it the user can obtain the
job configuration resultant, so the users have a tool end-to-end.

The work presented here contributes to the establishment a framework to automate
Hadoop job configuration, through the following proposals:
\begin{itemize}
	\item a interface for users basead on domain specific language;
	\item an algorithm to automate a good choice of jobs configuration;
	\item a method for sampling data on Hadoop clusters.
\end{itemize}

As measure of performance we used the latency time that the job led to
conclude. Furthermore, we intend to use other measures of performance such as
amount of intermidiate data generate, network usage and among others.

\section{Outline}

\begin{itemize}
	\item Chapter \ref{cha:background} introduces the fundamental concepts of the MapReduce framework.
    \item Chapter \ref{cha:bacAlg} presents the bacteriological algorithm.
    \item Chapter \ref{cha:sample} presents the method to generate sampling
    data.
    \item Chapter \ref{cha:dsl} introduces the concepts of the domain specific language.
	\item In chapter \ref{cha:proposal} we presents our framework with all components.
    \item In chapter \ref{cha:experiments} we discussed a case study performed with our solution.
	\item In chapter \ref{cha:conclusion} we conclude our results.
\end{itemize}

%The next section introduces the  fundamental concepts of the MapReduce framework and briefly describes different testing scenarios for MapReduce applications. Chapter~\ref{cha:proposal} introduces two test quality assessment methods, describes methods to data test generation and meta-heuristic search methods. Two data test generators are presented with their results.
%Chapter~\ref{cha:related} discusses related work. 
%Chapter~\ref{cha:conclusion} concludes.

% chapter introduction (end)



\chapter{Key-value model, MapReduce and Hadoop} % (fold)
\label{cha:background}

This chapter introduces some concepts that are used in the subsequent sections:
the key-value model, MapReduce paradigm and Hadoop framework.

\section{Key-value model}\label{section:mapreduce}

Key-value model is a simplified model for data storage. It is based on one linked
pair: the key and the value. Generally the pair is stored pure without aggregation or
any creation of data schema, thus all detailing of data is done in runtime. Unlike
other models such as relational model~\cite{codd:1970} in which the simplistic notion
of relation already gives some sense for the data, similarly to the hierarchical
data model~\cite{silber:2005} in which the links that connect the records give
details for the data.

To solve some particularities involving relational model was created data warehouse,
it is a repository that aggregates data from several sources ~\cite{silber:2005},
to make such aggregations is used the technique Extract Transform Load(\textbf{ETL}),
the data are extracted from sources, so they are transformed and load in a data
warehouse.

Due the simple storage of the data in key-value model the data \textit{transformation}
is done in the last fase, in other words, the data make sense when they are required,
there is a inversion of ETL to ELT(Extract Load Transform), but such inversion
cause one trouble to process a large amount of data, thus one big computing
power is necessary to query the data. One programming paradigm that handles the
key-value model is the MapReduce, it is going to present in the next section.

\section{MapReduce}\label{section:mapreduce}
MapReduce is a programming system for high-level that allow many processes of one
database can be written in simple way, according by Molina~\cite{molina:2009}.
That database processes aim to process a large amount of data that are splitted
and assigned to set computers, called computers cluster. Thus can improve the
performance obtained by parallelism, omitting all complexity for that the user
focus stays in the main problem that is the data processing.

The MapReduce paradigm have been implemented under key-value model, it was created
to process a large amount of data and benefits from data parallelism, consequently
builds large-scale parallel data processing applications. The paradigm is inspired
on the high-level \textbf{Map} and \textbf{Reduce} primitives from functional
programming languages. Hence the programmers can focus only in creation of the two
higher-order funcions to solve a specific problem and to generate the necessary
data, so it can just define the precice behavior of those functions.

Acording by\cite{dean:2008}: "the computation takes a set of \textit{input}
key/value pairs, and produces a set of \textit{output} key/value pairs.". The user
write the map function that receives a set of input key/value pair and produces an
\textit{intermidiate} set of key/value pairs. The reduce function written for the
user receives the intermidiate set as input and produces the \textit{resultant}
set of key/value pairs. This process is shown in Figure~\ref{fig:mapReduce}:

\begin{figure}[htbp]
	\centering
	\includegraphics[width=\columnwidth]{img/mapReduce.jpg}
	\caption{Map and Reduce process.}\label{fig:mapReduce}
\end{figure}

\section{Hadoop}
Map and reduce functions are present in Lisp and others functional languages. Recently
the MapReduce paradigm have been implemented by several frameworks such as Greenplum
MapReduce~\cite{Greenplum:2008}, Aster Data~\cite{Aster:2011}, Nokia
Disco~\cite{Mundkur:2011}, Microsoft Dryad~\cite{Isard:2007}, among others. One
open-source implemantation is the Hadoop that is a framework for reliable,
scalable, distributed computing~\cite{hadoop}.

The Hadoop provide an interface to implement the map and reduce functions in high-level.
It was projected for the user focus just on the implementation those functions,
without worrying with the issues involving the distributed computing. All aspects
involving the distributed computing and storage are left to the framework such as
split files, replication, fault tolerance, tasks distribuition etc.

There are two main components on Hadoop:
\begin{itemize}
	\item Hadoop Distributed File System(HDFS);
	\item Engine of MapReduce.
\end{itemize}

The HDFS stores all files in blocks, the block size is configurable per file, all
blocks of one file have the same block size except the last block. It is divide
in two components the \textit{NameNode} and \textit{DataNode}. The NameNode is placed
in one master machine, it store all metedatas and manages all DataNodes, any aspect
involving distributed storage is responsible by this component. The DataNode stores
the data, when one DataNode starts it connects to NameNode, then responds to requests
from the NameNode for filesystem operations.

The engine of MapReduce is responsible by the parallel processing, it is constituted
by one master machine and a lot of slave machines, also called workers. The master
designates which slaves will receive map and reduce tasks with its respective input
blocks. The worker who receive a map task is called mapper and the slave who receive
reduce task is called reducer. All aspects involving the distributed computing
management is responsible by the master like mappers failure, reducers failure,
scheduling tasks, shuffling intermediate files etc.

\subsection{Job processing}

A job is a program in high-level languages(java, ruby or python) that implements the
map and reduce functions. The master machine receive job submission with the relative
input directory in the HDFS where are all files to process. This files must be
inserted previously in the HDFS. Then the master requests to the NameNode infomation
about the blocks and file locations, after that it deploys copies of the job across
several workers.

With the blocks information acquired the map task is scheduled to a set of workers
with its respective input blocks, then the mappers process each input blocks, 
generate key/value intermediate pairs and append its in intermediate files, when
the mapper instance terminate it notify the master. The master splited the intermediate
files in blocks and shuffled it to the reducers to process, when all reducers
intances terminate, they append their result to the final output file. The data
flow between mappers and reducers are shown in Figure~\ref{fig:mrexecute}.


%%Fazer uma nova figura demonstrando o que foi escrito acima
\begin{figure}[htbp]
	\centering
	\includegraphics[width=\columnwidth]{img/mapreduce-en.pdf}
%    \includegraphics[bb=0 0 1280 960]{img/mapreduce-en.pdf}
	\caption{Execution of Map and Reduce operations}\label{fig:mrexecute}
\end{figure}

The whole processing is based on \tuple{key,value} pairs. The mappers receive the
file blocks, the mappers call the map function and pass the line number as key and
the line as the value, so the pair "line number/line contente" is the \tuple{k1,v1}.
The map generate the intermediate result set of key and values \tuple{set(k2,v2)}, 
when the mappers finished all values for \textit{k2} are agrouped in a list and
the respective pair \tuple{k2, list(v2)} is generated. This pairs are sorted and
pass as input for reducers that generate the result set:

\begin{center}
\begin{tabular}{c c c c}
	\hline
	   map & $k1,v1$ & $\rightarrow$  & $set(k2, v2)$ \\
	   reduce & $k2, list(v2)$ & $\rightarrow$ & $set(v2)$ \\
	\hline
\end{tabular}
\end{center}

Eventually, when the map result are already available in memory, a local reduce
function \emph{Combiner} is used for optimization reasons, then all values for
determinated key are combined, resulting in a local set \tuple{k2, list(v2)}.
This function runs after the Map and before the Reduce functions and is run on
every node that run map functions. The Combiner may be seen as a \emph{mini-reduce}
function, which operates only on data generated by one machine.

A good example of a MapReduce job is the Grep application, which receives as an
input several textual documents and as an output a set of pairs \tuple{Key,Value},
where each key is a different pattern found and the value is the number of occurrences 
of the pattern in the files. The responsibility of the Mapper is to find pattern
in the files and the reduce is to sum the amount found each patterns.

The Java implementation of the map function is presented in Listing~\ref{listing:mapper}. 
The \code{map()} method has four parameters: \code{key}, which is never used; \code{value},
one line that contains the text to be processed; the \code{output}, which will receive
the output pairs and \code{reporter} for debug. The body of the method uses the class
\code{Pattern} to describe a desired pattern, the class \code{Matcher} to find this
pattern, when pattern are found the pair \tuple{matching, 1} is emited to
output.

\singlespacing
\begin{listing}[H]
\begin{minted}[frame=lines,framesep=2mm,fontfamily=courier,fontsize=\scriptsize]{java}
public class RegexMapper<K> extends MapReduceBase
			implements Mapper<K, Text, Text, LongWritable> {

    private Pattern pattern;
    private int group;

    public void configure(JobConf job)
    {
        pattern = Pattern.compile(job.get("mapred.mapper.regex"));
    }

    public void map(K key, Text value, OutputCollector<Text, LongWritable> output,
					Reporter reporter) throws IOException {
        String text = value.toString();
        Matcher matcher = pattern.matcher(text);
        while (matcher.find())
        {
            output.collect(new Text(matcher.group()), new LongWritable(1));
        }
    }
}
\end{minted}
\caption{Class RegexMapper packed in Hadoop~\cite{hadoop}} 
\label{listing:mapper}
\end{listing}

\doublespacing
The implementation of the reduce function is presented in Listing~\ref{listing:reducer}.
The \code{reduce()} method has also four parameters: \code{key}, which contains
a single matching string; \code{values}, a set containing all values associated
to the key (i.e. the matching); \code{output pair}, the resultant pair \tuple{matching,
total} and \code{reporter} for debug. The behavior of the method is quite simple,
it sums all values associated to the key and then writes a pair containing the same
key and the total of matching found.
\singlespacing
\begin{listing}[H]
\begin{minted}[frame=lines,framesep=2mm,fontfamily=courier,fontsize=\scriptsize]{java}
public class LongSumReducer<K> extends MapReduceBase
			 implements Reducer<K, LongWritable, K, LongWritable> {

    public void reduce(K key, Iterator<LongWritable> values,
                     OutputCollector<K, LongWritable> output, Reporter reporter)
                throws IOException {

    // sum all values for this key
    long sum = 0;
    while (values.hasNext())
    {
        sum += values.next().get();
    }

    // output sum
    output.collect(key, new LongWritable(sum));
  }

}
\end{minted}
\caption{Class LongSumReducer packed in Hadoop~\cite{hadoop}} 
\label{listing:reducer}
\end{listing}

\doublespacing
An example of the inputs and the outputs of both functions when applied to a
simple sentence is presented in Table~\ref{table:regexp}. We aplied the following
regular expression: \\ \code{"[a-z]$*$o[a-z]$*$"}, this expression find the words that
contains the vowel \code{o} in the midle of them.

\begin{table}[H]
	\begin{center}
	\begin{tabular}{c p{.4\columnwidth} c p{.3\columnwidth} }
		\hline
		map & "Test for hadoop regular expression inside hadoop" & $\rightarrow$ & \tuple{for,1},\tuple{hadoop,1}, \tuple{expression,1}, \tuple{hadoop,1} \\
		reduce & \tuple{for,\{1\}}, \tuple{hadoop,\{1,1\}}, \tuple{expression,\{1\}} & $\rightarrow$ & \tuple{for,1},\tuple{hadoop,2}, \tuple{expression,1}\\
		\hline
	\end{tabular}
	\end{center}
	\caption{Regular expression example}
	\label{table:regexp}
\end{table}

% chapter chapter_name (end)

\chapter{Domain-Spcific Language} % (fold)
\label{cha:dls}

A \textit{domain-specific language} (DSL) is way to approach of some specific
context through appropriate notations and abstractions~\cite{deursen:2000}. DSL
transforms a particular problem domain into a context intelligible for expert
users that can work in a familiar environment.

Problem domain is a crucial term of DSL that requires prior background of the
developers in the specific context, so the developers must be expert in the domain
in order to develop DSLs that cover all features required for the users. There are
a lot of examples of DSLs in differents domains, \textbf{(LEX, YACC, Make, SQL,
HTML, CSS, LATEX, etc.)} are classical examples of DSLs~\cite{bentley:1986}.

DSLs are usually focused in its domains containg notations and specific abstractions,
normally DSLs are \textit{small} and \textit{declarative} languages. However, a
DSL can be extended to others	 domains, in this case such DSL is
general-purpose language (GPL), because its expressive power is not restricts
an exclusive domain, examples of such DSLs are \textbf{Cobol and Fortran}, which
could be viewed as languages focused towards the domain of business and scientific
programming  ~\cite{deursen:2000}, respectively, but they are not restricts just
in this domains.

DSL are used in several big areas, such \textbf{Software Engineering}, 
\textbf{Artificial Intelligence}, \textbf{Computers Architecture}(in this area a
good exemple is VHSIC Hardware Description Language (VHDL), where VHSIC mean 
{\bf V}ery {\bf H}igh {\bf S}peed {\bf I}ntegrated {\bf C}ircuits), \textbf{Database
Systems}(SQL is a classical example already cited), \textbf{Network}(where its
protocols are examples of DSLs), \textbf{Distribuited Systems}, \textbf{Multi-Media}
and among others. A current area that have been emerged recently is \textbf{Big Data}
, this area may be considered as a sub area of Database, but is has many
particularities that involve a mix features of Database and Distributed Systems.

%\input{capitulo4.tex}
%\input{capitulo5.tex}
%\input{capitulo6.tex}
%\input{anexo1.tex}     % se houver anexo



\bibliographystyle{brazil}
\bibliography{kepe}
% utilize macros (3 primeiras letras do mes em ingles, minusculas) no seu
% .bib para atribuir o nome do mes em portugues nas referencia,
% se o style for brazil, outros estilos tambem aceitam estas macros
% Ex:
%
% @InProceedings{teste,
%   author =       {Luciano}
%   year =         {2000}
%   month =        {}#sep;
% }
%
\addcontentsline{toc}{chapter}{\MakeUppercase{Bibliografia}}

\singlespacing
\makecapadissertacao

\end{document}
