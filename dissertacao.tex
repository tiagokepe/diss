\documentclass[12pt,a4paper]{ufpr}

% \usepackage[portuges,brazil]{babel}
% \usepackage[portuguese,brazil]{babel}

\usepackage[english]{babel}
\usepackage[utf8]{inputenc}
\usepackage{amssymb,amsmath,amsfonts}
\usepackage{epsfig}
\usepackage{multirow}


\usepackage{ifthen,graphicx,color}
%\usepackage{isolatin1}
\usepackage{amssymb}
\usepackage{subfigure}
\usepackage{caption2}
\usepackage{setspace}
\usepackage{ps-macros}
% \usepackage{psfig}

\usepackage{indentfirst}

\setcounter{secnumdepth}{3}    % n - numero de niveis de subsubsection numeradas
\setcounter{tocdepth}{3}       % coloca ate o nivel n no sumario

\title{KonfigJob: A framework based in bacteriological algorithm for Hadoop
job configuration}
\author{Tiago Rodrigo Kepe}
\advisortitle{Advisor} % ou Orientador
\advisorname{Prof. Dr. Eduardo C. de Almeida}
\advisorplace{Informatics Department, Federal University of Paran�, Brazil}
\city{Curitiba}
\year{2013}

\banca        % nao insira o nome do orientador, ja eh feito automaticamente
{Prof. Dr. Marcos D. Del Fabro}{Informatics Department, Federal University of Paran�}
{Prof. Dr. Gerson Suny�}{Informatics Department, INRIA - University of Nantes, France}
{}{}
{}{}    % se houver um quarto membro na banca, inserir nome e instituicao

\defesa{04 de outubro de 2000} % dia em que foi realizada a defesa da dissertacao


\begin{document}

\makecapaproposta             % cria capa para proposta%
%makecapadissertacao           % cria capa para dissertacao de mestrado %
%makerosto                     % cria folha de rosto para versao final da UFPR %
%\maketermo                     % cria folha com o termo de aprovacao da dissertacao%

%\singlespacing           % espacamento 1 - capa UFPR%
%\onehalfspacing          % espacamento 1/2 %
\doublespacing            % espacamento 2 - UFPR %

\pagestyle{headings}
\pagenumbering{roman}

%\chapter*{Agradecimentos}
%\input{agradecimentos.tex}          % possiu somente o texto

\tableofcontents

%\listoffigures         % se houver mais do que 3 figuras
%\addcontentsline{toc}{chapter}{\MakeUppercase{Lista de Figuras}}
%\newpage

%\listoftables        % se houver mais do que 3 tabelas
%\addcontentsline{toc}{chapter}{\MakeUppercase{Lista de Tabelas}}
%\newpage

\chapter*{Resumo}
\addcontentsline{toc}{chapter}{\MakeUppercase{Resumo}}
Texto do resumo....
           % somente o texto
\newpage

\chapter*{Abstract}
\addcontentsline{toc}{chapter}{\MakeUppercase{Abstract}}
% abstract
Currently with the popularity of internet and phenomenon of the social
networks a large amount of data is generated day-to-day. MapReduce appears as a
powerful paradigm to analyse and process such amount of data. The Hadoop framework
implements the MapReduce paradigm, in which a simple interface is available to
implement MapReduce(MR) jobs. However, in MR jobs developers are allowed to setup several
parameters to draw optimal performance from the available resources, but find a
good configuration is time consuming and a configuration found in an execution
may be impracticable for the next time.

In order to facilitate and automate tuning hadoop jobs, we propose a self-tuning
based on data sampling. Our approach allows to find a good job configuration considering
the data stored and the job in question. The users till can provide their usual
job configurations then get the new job configuration that will be more appropriate
with the current state of data stored and the hadoop cluster. So the users have
end-to-end tool to automate the choice of knobs for each job.
        % somente o texto
\newpage


\pagenumbering{arabic}

\chapter{Introduction} % (fold)
\label{cha:introduction}

\section{Motivation}

MapReduce~\cite{Dean:2004} became the industry de facto standard for parallel processing. Attractive features such as scalability and reliability motivate many large companies such as Facebook, Google, Yahoo and research institutes to adopt this new programming paradigm. 
These organizations rely on Hadoop~\cite{White:2009}, an open-source implementation of MapReduce, to process their information.
Besides Hadoop, several other implementations are available: Greenplum MapReduce~\cite{Greenplum:2008}, Aster Data~\cite{Aster:2011}, Nokia Disco~\cite{Mundkur:2011},  Microsoft Dryad~\cite{Isard:2007}, among others. 

MapReduce has a simplified programming model, where data processing algorithms are implemented as instances of two higher-order functions: Map and Reduce. All complex issues related to distributed processing, such as scalability, data distribution and reconciliation, concurrence, fault tolerance, etc., are managed by the framework. 
The main complexity that is left to the developer of a MapReduce-based application (also called a job) lies in the design decisions made to split the application specific algorithm into two higher-order functions. Even if some decisions may result in a functionally correct application, bad design choices might also lead to poor resource usage.

%%%%%%%%%%%%%%%%%%%% Reescrever este paragrafo
As for any other piece of software, testing may be used to evaluate the quality of MapReduce jobs. The efficiency of testing to detect quality issues in the jobs is in turn directly related to the quality of the test suite. MapReduce jobs work with large amounts of data, which is a major difficulty to generate relevant test data that can reveal quality issues. If on one hand using large data sets as input data is not scalable, on the other hand a small amount of data may not expose errors related to the large-scale aspects: efficient resource usage, correct merge of data, etc.

\section{Contribution}

We present an original approach for the generation of test data that target resource usage defects in MapReduce jobs. We use an evolutionary algorithm to generate test data and propose semantic mutation operators to evaluate the quality of the data with respect to their ability at detecting issues in the design of the map and reduce functions. This focus on testing design decisions lies on two observations. First, these design decisions are essential to implement MapReduce jobs, and they should be systematically tested along with the functionality of the job. Second, the functionality of map and reduce is usually simple (small functions, simple control and data flow) since most of the complexity of these jobs (distribution, synchronization, etc.) is handled by the framework. Consequently, simple test cases tend to cover 100\% of the job under test, can be good at detecting algorithmic errors, but are not sufficient to test the efficiency of the map and reduce design. 

The work presented here contributes to the establishment of systematics testing techniques for MapReduce jobs, through the following proposals:
\begin{itemize}
	\item fault models that focus on design issues when splitting a task between map and reduce functions
	\item an automatic search-based technique to generate test data which target these faults
	\item a series of initial experiments that illustrate the difficulty of detecting these faults, and the capacity of our algorithm at generating data that target them
\end{itemize} 

\section{Outline}

\begin{itemize}
	\item Chapter \ref{cha:background} introduces the  fundamental concepts of the MapReduce framework.
	\item Chapter \ref{cha:testgen} introduces two test quality assessment methods, describes methods to data test generation and 
	      meta-heuristic search methods. Two data test generators are presented with their results.
	\item Chapter \ref{cha:proposal} presents our approach to test data generation for testing MapReduce systems.
	\item In chapter \ref{cha:experiments} we and discuss a case study performed with our solution.
	\item In chapter \ref{cha:conclusion} we conclude our results.
\end{itemize}

%The next section introduces the  fundamental concepts of the MapReduce framework and briefly describes different testing scenarios for MapReduce applications. Chapter~\ref{cha:proposal} introduces two test quality assessment methods, describes methods to data test generation and meta-heuristic search methods. Two data test generators are presented with their results.
%Chapter~\ref{cha:related} discusses related work. 
%Chapter~\ref{cha:conclusion} concludes.

% chapter introduction (end)

%\input{capitulo2.tex}
%\input{capitulo3.tex}
%\input{capitulo4.tex}
%\input{capitulo5.tex}
%\input{capitulo6.tex}
%\input{anexo1.tex}     % se houver anexo



\bibliographystyle{brazil}
\bibliography{kepe}
% utilize macros (3 primeiras letras do mes em ingles, minusculas) no seu
% .bib para atribuir o nome do mes em portugues nas referencia,
% se o style for brazil, outros estilos tambem aceitam estas macros
% Ex:
%
% @InProceedings{teste,
%   author =       {Luciano}
%   year =         {2000}
%   month =        {}#sep;
% }
%
\addcontentsline{toc}{chapter}{\MakeUppercase{Bibliografia}}

\singlespacing
\makecapadissertacao

\end{document}
