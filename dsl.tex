\chapter{Domain-Spcific Language} % (fold)
\label{cha:dls}

A \textit{domain-specific language} (DSL) is way to approach of some specific
context through appropriate notations and abstractions~\cite{deursen:2000}. DSL
transforms a particular problem domain into a context intelligible for expert
users that can work in a familiar environment.

Problem domain is a crucial term of DSL that requires prior background of the
developers in the specific context, so the developers must be expert in the domain
in order to develop DSLs that cover all features required for the users. There are
a lot of examples of DSLs in differents domains, \textbf{(LEX, YACC, Make, SQL,
HTML, CSS, LATEX, etc.)} are classical examples of DSLs~\cite{bentley:1986}.

DSLs are usually focused in its domains containg notations and specific abstractions,
normally DSLs are \textit{small} and \textit{declarative} languages. However, a
DSL can be extended to others	 domains, in this case such DSL is
general-purpose language (GPL), because its expressive power is not restricts
an exclusive domain, examples of such DSLs are \textbf{Cobol and Fortran}, which
could be viewed as languages focused towards the domain of business and scientific
programming  ~\cite{deursen:2000}, respectively, but they are not restricts just
in this domains.

DSL are used in several big areas, such \textbf{Software Engineering}, 
\textbf{Artificial Intelligence}, \textbf{Computers Architecture}(in this area a
good exemple is VHSIC Hardware Description Language (VHDL), where VHSIC mean 
{\bf V}ery {\bf H}igh {\bf S}peed {\bf I}ntegrated {\bf C}ircuits), \textbf{Database
Systems}(SQL is a classical example already cited), \textbf{Network}(where its
protocols are examples of DSLs), \textbf{Distribuited Systems}, \textbf{Multi-Media}
and among others. A current area that have been emerged recently is \textbf{Big Data}
, this area may be considered as a sub area of Database, but is has many
particularities that involve a mix features of Database and Distributed Systems.
